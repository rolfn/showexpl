
% Rolf Niepraschk, 2020-05-03, Rolf.Niepraschk@gmx.de

\listfiles\errorcontextlines=100
\documentclass[a4paper,draft,twoside]{article}

\usepackage[dvipsnames]{xcolor}
\usepackage[textwidth=12cm,marginparwidth=4cm]{geometry}
\usepackage[final]{showexpl}

\usepackage{iftex}

\iftutex % LuaTeX, XeTeX
  \usepackage{fontspec}
  \setmonofont{DejaVuSansMono}[Scale=MatchUppercase]
\else % old engines
  \usepackage[T1]{fontenc}
  \usepackage{lmodern}
  \usepackage[scaled]{DejaVuSansMono}
\fi

\lstset{%
  basicstyle=\ttfamily\small,
  commentstyle=\itshape\ttfamily\small,
  showspaces=false,
  showstringspaces=false,
  breaklines=true,
  backgroundcolor=\color{lightgray},
  breakautoindent=true,
  captionpos=t
}

\iftrue
\lstset{explpreset={numbers=left,numberstyle=\tiny,numbersep=.3em,
  xleftmargin=1em,columns=flexible,language={}}}
\fi

\usepackage[font=small,labelfont=bf,justification=raggedright,%
  singlelinecheck=false]{caption}
\usepackage{amsmath}

\AtBeginDocument{%
  \renewcommand*\lstlistlistingname{Examples}
  \renewcommand*\lstlistingname{Example}
}

\newcommand*{\MARKER}%
  {\noindent\strut\vrule
   \hrulefill~half text area~\hrulefill\vrule
   \hrulefill~half text area~\hrulefill\vrule
    \marginpar{\strut\vrule\hrulefill~margin area~\hrulefill\vrule}}
    
\begin{filecontents*}{ex1.tex}
Line  1 \par
Line  2 \par
Line  3 \par
Line  4 \par
Line  5 \par
Line  6 \par
Line  7 \par
Line  8 \par
Line  9 \par
Line 10 \par
Line 11 \par
Line 12 \par
Line 13
\end{filecontents*}

\begin{document}

\section*{The \LaTeX\ package \textsf{showexpl}}

\bigskip

\lstlistoflistings

\subsection*{The \textsf{listings} parameters still works}

\begin{LTXexample}[pos=t,numbers=none,
  codefile=\jobname-\theltxexample.tex]
\Large\LaTeX{} \LaTeX{}
\LaTeX{} \LaTeX{}
\end{LTXexample}

\MARKER

\subsection*{The \texttt{pos}, \texttt{overhang}, and
  \texttt{caption} parameters}

\begin{LTXexample}[pos=b,overhang=.5\marginparwidth+.5\marginparsep,%
  caption={The \texttt{overhang} parameter}]
\Large\LaTeX{} \LaTeX{}
\LaTeX{} \LaTeX{}
\end{LTXexample}
  
\MARKER

\begin{LTXexample}[pos=l,hsep=50pt,width=55mm]
\Large\LaTeX{} \LaTeX{}
\LaTeX{} \LaTeX{}
\end{LTXexample}

\MARKER

\subsection*{The \texttt{wide} parameter with inner and outer position}

\begin{LTXexample}[pos=o,wide,caption={The \texttt{wide} parameter},%
  label=ex:Gustav]
\Large\LaTeX{} \LaTeX{}
\LaTeX{} \LaTeX{}
\end{LTXexample}
  
\MARKER

\begin{LTXexample}[pos=i,wide]
\Large\LaTeX{} \LaTeX{}
\LaTeX{} \LaTeX{}
\end{LTXexample}


%--------------
\newpage

\subsection*{More examples on an even (left) page}

\begin{LTXexample}[pos=t]
\Large\LaTeX{} \LaTeX{}
\LaTeX{} \LaTeX{}
\end{LTXexample}

\MARKER

\begin{LTXexample}[pos=b,overhang=.5\marginparwidth+.5\marginparsep,%
  captionpos=b,caption={The \texttt{overhang} parameter again},%
  label=ex:Fridolin]
\Large\LaTeX{} \LaTeX{}
\LaTeX{} \LaTeX{}
\end{LTXexample}
  
\MARKER

\begin{LTXexample}[pos=l,hsep=50pt,width=55mm]
\Large\LaTeX{} \LaTeX{}
\LaTeX{} \LaTeX{}
\end{LTXexample}

\MARKER

\begin{LTXexample}[pos=o,wide,captionpos=b,%
  caption={The \texttt{wide} parameter again}]
\Large\LaTeX{} \LaTeX{}
\LaTeX{} \LaTeX{}
\end{LTXexample}
  
\MARKER

\begin{LTXexample}[pos=i,wide]
\Large\LaTeX{} \LaTeX{}
\LaTeX{} \LaTeX{}
\end{LTXexample}

  
%--------------
\newpage

\subsection*{Whole \LaTeX{} documents as example code and 
  the para\-meters \texttt{preset}, \texttt{rframe}, and \texttt{rangeaccept}}
  
\begin{LTXexample}[pos=r,wide,width=.65,preset=\LARGE,rframe={}]
\documentclass[a4paper,twoside]{article}
\begin{document}
  \begin{equation}
    \sigma(t)=\frac{1}{\sqrt{2\pi}}
    \int^t_0 e^{-x^2/2} dx 
  \end{equation}
\end{document}
\end{LTXexample}

\MARKER

\begin{LTXexample}[wide,width=.75,preset=\footnotesize,rframe=single,%
  codefile=\jobname-\theltxexample.tex]
\documentclass[a4paper,twoside]{article}
\usepackage{amsmath}
% enhancements for mathematical formulas
\begin{document}
\begin{equation}\label{eq:barwq}
\begin{split}
  H_c&=\frac{1}{2n} 
  \sum^n_{l=0}(-1)^{l}(n-{l})^{p-2}
  \sum_{l _1+\dots+ l _p=l}\prod^p_{i=1} 
  \binom{n_i}{l _i}\\
  &\quad\cdot[(n-l )-(n_i-l _i)]^{n_i-l _i}\cdot
  \Bigl[(n-l )^2-\sum^p_{j=1}(n_i-l _i)^2\Bigr].
\end{split}
\end{equation}
\end{document}
\end{LTXexample}

\LTXinputExample[%
  pos=r,backgroundcolor=\color{Goldenrod!50},%
  %firstline=3,lastline=6,%
  linerange={3-6,8-10},%
  float=tb,rangeaccept=true,%
  caption={[Floating Example]%
    This is a floating Example (parameter \texttt{rangeaccept=true})}]{ex1}

\newpage

\MARKER

\subsection*{Using a graphic as the result}

\begin{LTXexample}[pos=i,wide]
\Large\LaTeX{} \LaTeX{}
\LaTeX{} \LaTeX{}
\end{LTXexample}

\begin{LTXexample}[pos=i,wide,rframe={},%
graphic=result-picture]% graphic=\jobname-\theltxexample
\Large\LaTeX{} \LaTeX{}
\LaTeX{} \LaTeX{}
\end{LTXexample}

\begin{LTXexample}[caption={The \texttt{graphic} parameter},captionpos=b,%
  pos=i,wide,graphic={[width=.5\linewidth]result-picture}] 
\Large\LaTeX{} \LaTeX{}
\LaTeX{} \LaTeX{}
\end{LTXexample}

\newpage

\subsection*{The parameter \texttt{varwidth}}

\bigskip

\MARKER

\begin{LTXexample}[caption={Fix width of the result 
  (side-by-side default: \texttt{0.5\textbackslash linewidth})},%
  captionpos=b]
\setlength{\unitlength}{1cm}
\begin{picture}(2,2) \thicklines
  \thicklines
  \put(0,0){\line(1,1){2}}
  \put(0,2){\line(1,-1){2}}
\end{picture}
\end{LTXexample} 

\medskip

\begin{LTXexample}[caption={[The \texttt{varwidth} parameter]%
  Width of the result reduced to the ``natural'' 
  width (\texttt{varwidth=true})},varwidth,captionpos=b]
\setlength{\unitlength}{1cm}
\begin{picture}(2,2) \thicklines
  \put(0,0){\line(1,1){2}}
  \put(0,2){\line(1,-1){2}}
\end{picture}
\end{LTXexample}

\medskip

\begin{LTXexample}[caption={Fix width of the result 
  (default: \texttt{\textbackslash linewidth})},%
  pos=t,captionpos=b]
\setlength{\unitlength}{1cm}
\begin{picture}(2,2) \thicklines
  \put(0,0){\line(1,1){2}}
  \put(0,2){\line(1,-1){2}}
\end{picture}
\end{LTXexample}

\medskip

\begin{LTXexample}[caption={[The \texttt{justification} parameter]%
  Result is centered  (\texttt{varwidth=true})},%
  varwidth,captionpos=b,pos=t,justification=\centering]
\setlength{\unitlength}{1cm}
\begin{picture}(2,2)
  \thicklines
  \put(0,0){\line(1,1){2}}
  \put(0,2){\line(1,-1){2}}
\end{picture}
\end{LTXexample}

\end{document}
